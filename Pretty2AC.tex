\documentclass{article}
\title{Designate Fracking Wastewater 2AC}
\date{March 6, 2020}
\author{Phineas Greene and LJ Oby}

\usepackage{soul}
\usepackage{url}
\usepackage{booktabs}
\usepackage[colorlinks]{hyperref}
\hypersetup{citecolor=black}
\hypersetup{linkcolor=black}
\hypersetup{urlcolor=blue}
\setuldepth{o}

\addtolength{\oddsidemargin}{-.875in}
\addtolength{\evensidemargin}{-.875in}
\addtolength{\textwidth}{1.75in}

\addtolength{\topmargin}{-.875in}
\addtolength{\textheight}{1.75in}

\begin{document}
\pagenumbering{gobble}
\maketitle
\newpage

\pagenumbering{arabic}
\tableofcontents
\newpage

\section{Opening Quotes}

\subsection{California Pits}
\paragraph{}
\small
\textit{
\underline{David Hasemyer, Inside Climate News, Nov 20, 2014}
  (InsideClimate News reporter David Hasemyer is co-author of the "Dilbit Disaster: Inside the Biggest Oil Spill You've Never Heard Of," which won the 2013 Pulitzer Prize for National Reporting, and co-authored the 2016 Pulitzer Prize finalist series "Exxon: The Road Not Taken." Prior to joining ICN, Hasemyer had an award-wining tenure at The San Diego Union-Tribune as an investigative reporter. Hasemyer's newspaper work has been recognized by the Associated Press, the Society for Professional Journalists, the Society of American Business Editors and Writers. He also has been a finalist for the Gerald Loeb Award and the Robert F. Kennedy Award for Social Justice and Human Rights. ) “Hazards of Open Pits for Storing Wastewater From Fracking Is Focus of New Study” Published by Inside Climate News
\url{https://insideclimatenews.org/news/20141120/hazards-open-pits-storing-wastewater-fracking-focus-new-study }}
\normalsize
\paragraph{}
``Unlined open-air wastewater pits brimming with the toxic leftovers of fracking and other types of oil and gas development are threatening California's air and water quality, according to a study by two national environmental organizations.''


\subsection{Visiting A Holding Site}
\paragraph{}
\small
\textit{  
\underline{Andrew Grinberg, Clean Water Action/Clean Water Fund, November 2014}
  (Andrew began working with Clean Water in 2006 and in 2016 joined the national program team in Washington DC. From 2011 through 2015, he managed the California oil and gas program, leading statewide efforts to rein in dirty oil and gas activities including disposal of wastewater into unlined pits, underground injection into drinking water sources and unregulated hydraulic fracturing. Andrew has also managed Clean Water Action field canvass programs in San Francisco, Austin, and DC. Andrew was born and raised in San Francisco, attended Occidental College and currently lives in DC.) “In the pits ” Published by Clean Water Action/Clean Water Fund 
\url{https://assets.documentcloud.org/documents/1362974/ca-oil-and-gas-pit-report.pdf} }
\normalsize
\paragraph{}
``Driving north along Highway 33  nicknamed the ``Petroleum Highway''  the group turned off to the east, down a narrow, unmarked and publicly accessible dirt road. A large tanker truck was visible in the distance. It appeared to be dumping water into the ground. Minutes later, the tour approached a gate with a sign reading ``Danger H2S May Be Present.'' Tour members stepped out of the vehicles and were immediately hit with a noxious odor. Several tied bandanas around their mouths and noses to block the fumes. In less than five minutes, many in the group complained of nausea and headaches. The site consisted of a few dozen long narrow ponds, some with standing liquids of different shades of green, brown and black, some dry and empty. The closest pond contained two thick pipes that were discharging steaming black and green fluids to the pond, while vapors visibly rose off the surface of the ponds. A thick black ring of what appeared to be oil rimmed the bank and a shimmering black layer floated on the surface. Pipes connected the first pond receiving the discharge to other, larger ponds stretching out hundreds of yards into the distance.''

\section{Inherency/Facts}

\subsection{Fracking Ponds Used}
\paragraph{}
\small
\textit{
\underline{Well Water Solutions, May 2018}
  (Well Water Solutions and Rentals Inc. was founded on the principals of creating strong business to business relationships. We have brought together an extremely talented team of business leaders with oil industry experience, to develop breakthrough solutions and unmatched business to business growth opportunities. We understand that business networking is a vital key to continued growth and business expansion worldwide and that this is also vital for a successful company.) “6 Ways We Can Make Your Frac Site More Environmentally Friendly” Published by Well Water Solutions  \url{https://wwstanks.com/2018/05/25/6-ways-we-can-make-your-frac-site-more-environmentally-friendly/ }}
\normalsize
\paragraph{}
\ul{``Frac ponds are another common reason why fracking companies have a bad reputation with environmentalists and the general population. Frac ponds are essentially open-air pools where wastewater is stored and allowed to evaporate into the air over time. This has obvious air contamination concerns, but these ponds are also dangerous for wildlife that confuse the dirty wastewater with fresh water to drink.} Our above-ground storage tanks are an excellent alternative to these frac ponds, and they are much more environmentally friendly.”

\subsection{Not Considered Hazerdous Waste}
\paragraph{}
\small
\textit{
\underline{Alexandra Zissu, Natural Resources Defense Council, January 27, 2016}
(Alexandra Zissu is the author of The Conscious Kitchen and coauthor of Planet Home, The Complete Organic Pregnancy, and The Butcher's Guide to Well-Raised Meat. She has written for The New York Times, New York magazine, Health, and Bon Appétit. A native New Yorker, she recently moved to the Hudson Valley with her family to be closer to the farms that feed them. You can find her at alexandrazissu.com.) “How to Tackle Fracking in Your Community” Published by Natural Resources Defense Council  \url{https://www.nrdc.org/stories/how-tackle-fracking-your-community}}
\normalsize
\paragraph{}
``\ul{Fracking enjoys loopholes from a number of our bedrock environmental laws,'' Raichel notes. For example, oil and gas waste is not considered a hazardous waste under the Resource Conservation and Recovery Act.} This can make it difficult for concerned citizens to push the needle on a federal level, but it’s still important to call your elected representatives and urge them to close these loopholes. ``Although sweeping change might be slow in coming, staying vocal keeps the pressure on elected officials and industry,” Raichel says. ``As we’ve seen before, if there is enough of a groundswell, it will make a difference.”

\subsection{Exempt From RCRA}
\paragraph{}
\small
\textit{
\underline{Elizabeth Ridlington and Kim Norman, Environment America Research and Policy Center, 2016}
(Elizabeth Ridlington is associate director and senior policy analyst with Frontier Group. She focuses primarily on global warming and clean vehicles and has written dozens of reports on these and other subjects. Her 2018 report Trouble in the Air found that millions of Americans regularly breathe air polluted by smog and particulate pollution, and was covered in the Washington Post, the Atlantic, and National Public Radio. Her report Moving America Forward assessed the impact of state and national policies to reduce global warming emissions, and she has written numerous reports on the benefits of state adoption of stronger vehicle emission standards. Elizabeth graduated with honors from Harvard with a degree in government. She joined Frontier Group in 2002. She lives in Northern California with her husband and son.) “Fracking by the Numbers” Published by Environment America 
\url{https://environmentamerica.org/sites/environment/files/reports/ Fracking\%20by\%20the\%20Numbers\%20vUS.pdf}}
\normalsize
\paragraph{}
``Waste from oil and gas fracking is exempt from the hazardous waste provisions of the federal Resource Conservation Recovery Act (RCRA), exacerbating the toxic threats posed by fracking wastewater. For other industries, the threats posed by toxic waste have been at least reduced due to the adoption of the RCRA, which provides a national framework for regulating hazardous waste. In other industries, illegal dumping is reduced by cradle-to-grave tracking and criminal penalties. Health-threatening practices such as open waste pits, disposal in ordinary landfills, and road spreading are prohibited.”

\section{Significance/Facts}

\subsection{Wastewater Used For Irrigation}
\paragraph{}
\small
\textit{
\underline{Environmental Working Group, EcoWatch, Oct. 27, 2016}
(The Environmental Working Group’s mission is to empower people to live healthier lives in a healthier environment. With breakthrough research and education, we drive consumer choice and civic action.) “Why Are California Farmers Irrigating Crops With Oil Wastewater?” Published by EcoWatch \url{https://www.ecowatch.com/california-crops-oil-wastewater-2064638069.html}}
\normalsize
\paragraph{}
``In the last three years, farmers in parts of California's Central Valley irrigated nearly 100,000 acres of food crops with billions of gallons of oil field wastewater possibly tainted with toxic chemicals, including chemicals that can cause cancer and reproductive harm, according to an Environmental Working Group (EWG) analysis of state data.”

\subsection{What's In The Wastewater}
\paragraph{}
\small
\textit{
\underline{Award-winning journalist Melissa Troutman, Truthout, February 21, 2019}
( Melissa Troutman is an award-winning journalist and filmmaker who joined Earthworks as a research and policy analyst in 2018. In 2011, she co-founded Public Herald, an investigative news outlet that produced the documentary films Triple Divide, Triple Divide [REDACTED] and INVISIBLE HAND.)``What is fracking and why is it controversial?'' Published by Truthout
\url{https://truthout.org/articles/is-drilling-and-fracking-waste-on-your-sidewalk-or-in-your-pool/}}
\normalsize
\subparagraph{}
``\textbf{Salt.} Flowback and produced water are highly salty. This is because salts are added to the fracturing fluid and also released from the geologic formation. Produced water is so famous for salinity that the hydrocarbon industry often refers to it simply as “saltwater” or “brine.” In the Marcellus Shale, flowback water has been measured at 32,300 mg per liter of sodium. For comparison, EPA guidelines call for a maximum of 20 mg/L in drinking water.
\subparagraph{}
\textbf{Industrial chemicals.} Flowback and produced water contain chemicals that have been injected into the well to facilitate drilling. For example, in the Marcellus Shale, flowback water contains high concentrations of sodium, magnesium, iron, barium, strontium, manganese, methanol, chloride, sulfate and other substances.
\subparagraph{}
\textbf{Hydrocarbons.} Produced water can contain hydrocarbons -- including the toxic substances benzene, toluene, ethylbenzene, and xylene – which can be freed during the drilling process.
\subparagraph{}
\textbf{Radioactive materials.}Water returned to the surface during drilling can carry naturally occurring radioactive materials, referred to by the industry as “NORM.” Flowback and produced water from several large U.S. shale formations have been found to contain the radioactive element radium. When produced water is salty and rich in chlorides, radium tends to be present in higher concentrations.
The EPA allows a maximum of 5 picocuries of radium per liter of drinking water. Produced water has been found to contain radium levels as high as 9,000 picocuries per liter (pCi/g)..”

\subsection{Wastewater Being Recycled}
\paragraph{}
\small
\textit{
\underline{Award-winning journalist Melissa Troutman, Truthout, February 21, 2019}
( Melissa Troutman is an award-winning journalist and filmmaker who joined Earthworks as a research and policy analyst in 2018. In 2011, she co-founded Public Herald, an investigative news outlet that produced the documentary films Triple Divide, Triple Divide [REDACTED] and INVISIBLE HAND.)``What is fracking and why is it controversial?'' Published by Truthout
\url{https://truthout.org/articles/is-drilling-and-fracking-waste-on-your-sidewalk-or-in-your-pool/}}
\normalsize
\paragraph{}
``Eureka Resources treats toxic wastewater from drilling and fracking from natural gas in Pennsylvania, and the agency that approves this is the Pennsylvania Department of Environmental Protection (PADEP). In June 2014, Eureka Engineer Jerel Bogdan wrote to PADEP to confirm what, exactly, the company had to do in order to sell the salt byproduct from the company’s waste treatment process to consumers.”

\subsection{States Need Federal Help}
\paragraph{}
\small
\textit{
\underline{Award-winning journalist Melissa Troutman, Truthout, February 21, 2019}
( Melissa Troutman is an award-winning journalist and filmmaker who joined Earthworks as a research and policy analyst in 2018. In 2011, she co-founded Public Herald, an investigative news outlet that produced the documentary films Triple Divide, Triple Divide [REDACTED] and INVISIBLE HAND.)``What is fracking and why is it controversial?'' Published by Truthout
\url{https://truthout.org/articles/is-drilling-and-fracking-waste-on-your-sidewalk-or-in-your-pool/}}
\normalsize
\paragraph{}
``You may be wondering how a company like Eureka could get away with repackaging byproducts from fracking and selling them without informing the public about what they really are. But the process is perfectly legal in Pennsylvania through a regulatory mechanism called “de-wasting.” De-wasting essentially means rebranding waste as a new product, often without significant treatment to remove health and environmental hazards.”

\subsection{Risk of Salt Contamination High}
\paragraph{}
\small
\textit{
\underline{Award-winning journalist Melissa Troutman, Truthout, February 21, 2019}
( Melissa Troutman is an award-winning journalist and filmmaker who joined Earthworks as a research and policy analyst in 2018. In 2011, she co-founded Public Herald, an investigative news outlet that produced the documentary films Triple Divide, Triple Divide [REDACTED] and INVISIBLE HAND.)``What is fracking and why is it controversial?'' Published by Truthout
\url{https://truthout.org/articles/is-drilling-and-fracking-waste-on-your-sidewalk-or-in-your-pool/}}
\normalsize
\paragraph{}
``Eureka tests its frack salt quarterly, but there is no testing of the product for radioactivity and other toxins on a more routine basis. This concerns Daniel Bain, a research professor at the University of Pittsburgh’s Department of Geology and Environmental Science who studies radioactivity. Bain told Public Herald that “all it takes is a little glitch in the process, and you can have a dirty salt at some point.”

\subsection{Analysis of Chemichals In Wastewater}
\paragraph{}
\small
\textit{
\underline{Infectious Disease specialist Judy Stone, Forbes, Feb 17, 2017}
(I am an Infectious Disease specialist and author of Resilience: One Family's Story of Hope and Triumph over Evil and of Conducting Clinical Research, the essential guide to the topic.) ``Fracking And What New EPA Means For Your Health'' Published by Forbes   
\url{https://www.forbes.com/sites/judystone/2017/02/17/fracking-and-what-new-epa-means-for-your-health/}}
\normalsize

\begin{table}[h!]
  \begin{center}
    \begin{tabular}{p{2.5cm} | p{1.7cm} | p{3cm} | p{3.5cm} | p{4.2cm}} % <-- Alignments: 1st column left, 2nd middle and 3rd right, with vertical lines in between
      \textbf{Chemical} & \textbf{Type of Additive} & \textbf{Why Used}& \textbf{Non-fracking Uses}& \textbf{Health Problems}\\

      \midrule
      Hydrochloric (muriatic acid) & Acid & helps dissolve rock, and make cracks & swimming pool chemical, toilet bowl cleaner & severe burns to skin, GI and respiratory tract\\
      \midrule
      Polyacrylamide & Reduces friction & minimizes friction in the pipes & water treatment, soil conditioner & nervous system damage, a carcinogen\\
      \midrule
      Methanol&Corrosion inhibitor&prevents corrosion and winterizing agent&used as a solvent and in biodiesel&wood alcohol - can cause blindness and death\\
      \midrule
      Ethylene glycol&Scale inhibitor&prevents scale in pipes&anti-freeze&poisonous\\
      \midrule
      Glutaraldehyde&Biocide&kills bacteria that might be corrosive to pipes&disinfecting medical equipment&commonly cause throat and lung irritation, and asthma\\
      \midrule
      Dimethyl-formamide&Corrosion inhibitor&prevents pipe corrosions&plastics&liver damage, high blood pressure\\
      \midrule
      Isopropanol&Surfactant&increases viscosity of the fluid&"rubbing alcohol," glass cleaner&contact irritation, headache, dizziness\\
      \midrule
      Ammonium persulfate&Breaker&delays breakdown of polymer chains&bleaching, plastics mfg.&respiratory distress, burning on contact

    \end{tabular}
  \end{center}
\end{table}
\end{document}